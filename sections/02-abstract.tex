\documentclass[../main.tex]{subfiles}
\begin{document}

\begin{abstract}
    A measurement of the mean lifetime $\tau_0$ of the muon has been performed, using secondary cosmic rays as a muon source. An experimental setup has been devised in order to select only muons stopping in dedicated absorbers, taking as measured time the time elapsed between the triggering from the passing muon and the detection of the decay daughter electron (or positron).
    
    The analysis led to an estimate of the muon lifetime
    \begin{equation*}
                \tau_0  =2.054\pm 0.055\, \si{\micro \second} 
    \end{equation*}
    which is compatible within 3 times the error with the currently acknowledged value.\\

    Also estimates for the atmospheric muon charge ratio $R$ and for the characteristic absorption time scales of negative muons $\tau_C$ have been obtained for both iron and concrete
    \begin{alignat*}{3}
        &R &=1.35\phantom{0}&\pm 0.18\phantom{0} \\
        &\tau_{C\,\textnormal{iron}} & =0.220&\pm 0.056\, \si{\micro \second} \\
        &\tau_{C\,\textnormal{concrete}} & =0.98\phantom{0}&\pm 0.30\phantom{0}\, \si{\micro \second}
    \end{alignat*}
    which are compatible within the error with the values expected for them.
\end{abstract}

\end{document}

\clearpage

