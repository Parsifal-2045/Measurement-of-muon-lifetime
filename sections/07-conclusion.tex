\documentclass[../main.tex]{subfiles}
\begin{document}
\section{Conclusions}

An apparatus for the measurement of muon lifetime has been devised. The system utilised 3 pairs of PMTs to select events in which cosmic ray muons stopped inside of a specific absorbing region. Two different absorbers were used, namely iron and concrete. Calibration working points were then chosen. Due to lack of experience this choice later revealed to be far from optimal, poisoning the acquired signals with unexpected stopping times. A possible reason for this behaviour was ascribed due to afterpulses though this explanation would require further investigation, especially for the dominating unexpected region below 40\,ns. In any case proper event selection led to the reduction of these spurious signals and fits were performed on the data finally leading to estimates for the muon lifetime $\tau_0$, the atmospheric muon charge ratio $R$ and the characteristic negative muon capture timescales $\tau_C$ for both iron and concrete absorbers.
\begin{alignat*}{3}
    &\tau_0 & =2.054&\pm 0.055\, \si{\micro \second} \\
    &R &=1.35\phantom{0}&\pm 0.18\phantom{0} \\
    &\tau_{C\textnormal{ iron}} & =0.220&\pm 0.056\, \si{\micro \second} \\
    &\tau_{C\textnormal{ concrete}} & =0.98\phantom{0}&\pm 0.30\phantom{0}\, \si{\micro \second}
\end{alignat*}
The muon lifetime is compatible within 3 times the error with the currently acknowledged value, while $R$ and $\tau_{C\textnormal{ iron}}$ are within one standard deviation. Also $\tau_{C\textnormal{ concrete}}$ is compatible with our estimate of the expected negative capture timescale, yet the lack of precise knowledge on the composition of the concrete makes it difficult to state something about the precision of this measurement.\\

The experiment has great margin of improvement. First of all, as already stated, a further and deep investigation of the PMTs signals and a better and thoughtful decision of working points may reduce, if not even completely remove, the unwanted signals observed on P1 and P2 chambers. Another point of improvement would certainly be the acquisition of a larger statistics especially for what concerns the background, which would require more than 2 days of data taking for becoming significant in order to improve the final results. Furthermore the system improvement that could introduce a magnetic field to discriminate positively and negatively charged muons would help mitigating the effects of negative muon capture but, in order to implement it, it is crucial to gain a deeper knowledge about the concrete absorber since an iron absorber could not be used in that case due to it being ferromagnetic. Knowledge improvement about the concrete absorber may also include a chemical analysis of the absorber itself.

\end{document}

\clearpage